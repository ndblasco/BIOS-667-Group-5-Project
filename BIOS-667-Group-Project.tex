% Options for packages loaded elsewhere
\PassOptionsToPackage{unicode}{hyperref}
\PassOptionsToPackage{hyphens}{url}
\documentclass[
]{article}
\usepackage{xcolor}
\usepackage[margin=1in]{geometry}
\usepackage{amsmath,amssymb}
\setcounter{secnumdepth}{-\maxdimen} % remove section numbering
\usepackage{iftex}
\ifPDFTeX
  \usepackage[T1]{fontenc}
  \usepackage[utf8]{inputenc}
  \usepackage{textcomp} % provide euro and other symbols
\else % if luatex or xetex
  \usepackage{unicode-math} % this also loads fontspec
  \defaultfontfeatures{Scale=MatchLowercase}
  \defaultfontfeatures[\rmfamily]{Ligatures=TeX,Scale=1}
\fi
\usepackage{lmodern}
\ifPDFTeX\else
  % xetex/luatex font selection
\fi
% Use upquote if available, for straight quotes in verbatim environments
\IfFileExists{upquote.sty}{\usepackage{upquote}}{}
\IfFileExists{microtype.sty}{% use microtype if available
  \usepackage[]{microtype}
  \UseMicrotypeSet[protrusion]{basicmath} % disable protrusion for tt fonts
}{}
\makeatletter
\@ifundefined{KOMAClassName}{% if non-KOMA class
  \IfFileExists{parskip.sty}{%
    \usepackage{parskip}
  }{% else
    \setlength{\parindent}{0pt}
    \setlength{\parskip}{6pt plus 2pt minus 1pt}}
}{% if KOMA class
  \KOMAoptions{parskip=half}}
\makeatother
\usepackage{longtable,booktabs,array}
\usepackage{calc} % for calculating minipage widths
% Correct order of tables after \paragraph or \subparagraph
\usepackage{etoolbox}
\makeatletter
\patchcmd\longtable{\par}{\if@noskipsec\mbox{}\fi\par}{}{}
\makeatother
% Allow footnotes in longtable head/foot
\IfFileExists{footnotehyper.sty}{\usepackage{footnotehyper}}{\usepackage{footnote}}
\makesavenoteenv{longtable}
\usepackage{graphicx}
\makeatletter
\newsavebox\pandoc@box
\newcommand*\pandocbounded[1]{% scales image to fit in text height/width
  \sbox\pandoc@box{#1}%
  \Gscale@div\@tempa{\textheight}{\dimexpr\ht\pandoc@box+\dp\pandoc@box\relax}%
  \Gscale@div\@tempb{\linewidth}{\wd\pandoc@box}%
  \ifdim\@tempb\p@<\@tempa\p@\let\@tempa\@tempb\fi% select the smaller of both
  \ifdim\@tempa\p@<\p@\scalebox{\@tempa}{\usebox\pandoc@box}%
  \else\usebox{\pandoc@box}%
  \fi%
}
% Set default figure placement to htbp
\def\fps@figure{htbp}
\makeatother
% definitions for citeproc citations
\NewDocumentCommand\citeproctext{}{}
\NewDocumentCommand\citeproc{mm}{%
  \begingroup\def\citeproctext{#2}\cite{#1}\endgroup}
\makeatletter
 % allow citations to break across lines
 \let\@cite@ofmt\@firstofone
 % avoid brackets around text for \cite:
 \def\@biblabel#1{}
 \def\@cite#1#2{{#1\if@tempswa , #2\fi}}
\makeatother
\newlength{\cslhangindent}
\setlength{\cslhangindent}{1.5em}
\newlength{\csllabelwidth}
\setlength{\csllabelwidth}{3em}
\newenvironment{CSLReferences}[2] % #1 hanging-indent, #2 entry-spacing
 {\begin{list}{}{%
  \setlength{\itemindent}{0pt}
  \setlength{\leftmargin}{0pt}
  \setlength{\parsep}{0pt}
  % turn on hanging indent if param 1 is 1
  \ifodd #1
   \setlength{\leftmargin}{\cslhangindent}
   \setlength{\itemindent}{-1\cslhangindent}
  \fi
  % set entry spacing
  \setlength{\itemsep}{#2\baselineskip}}}
 {\end{list}}
\usepackage{calc}
\newcommand{\CSLBlock}[1]{\hfill\break\parbox[t]{\linewidth}{\strut\ignorespaces#1\strut}}
\newcommand{\CSLLeftMargin}[1]{\parbox[t]{\csllabelwidth}{\strut#1\strut}}
\newcommand{\CSLRightInline}[1]{\parbox[t]{\linewidth - \csllabelwidth}{\strut#1\strut}}
\newcommand{\CSLIndent}[1]{\hspace{\cslhangindent}#1}
\setlength{\emergencystretch}{3em} % prevent overfull lines
\providecommand{\tightlist}{%
  \setlength{\itemsep}{0pt}\setlength{\parskip}{0pt}}
\usepackage{booktabs}
\usepackage{longtable}
\usepackage{array}
\usepackage{multirow}
\usepackage{wrapfig}
\usepackage{float}
\usepackage{colortbl}
\usepackage{pdflscape}
\usepackage{tabu}
\usepackage{threeparttable}
\usepackage{threeparttablex}
\usepackage[normalem]{ulem}
\usepackage{makecell}
\usepackage{xcolor}
\usepackage{caption}
\usepackage{graphicx}
\usepackage{siunitx}
\usepackage{hhline}
\usepackage{calc}
\usepackage{tabularx}
\usepackage{adjustbox}
\usepackage{hyperref}
\usepackage{bookmark}
\IfFileExists{xurl.sty}{\usepackage{xurl}}{} % add URL line breaks if available
\urlstyle{same}
\hypersetup{
  pdftitle={BIOS 667 Group 5},
  pdfauthor={Lily Bai; Nathalie Blasco; Yihao Peng; Lauren Rackley; Sirui Wu},
  hidelinks,
  pdfcreator={LaTeX via pandoc}}

\title{BIOS 667 Group 5}
\author{Lily Bai \and Nathalie Blasco \and Yihao Peng \and Lauren
Rackley \and Sirui Wu}
\date{2025-11-19}

\begin{document}
\maketitle

 
  \providecommand{\huxb}[2]{\arrayrulecolor[RGB]{#1}\global\arrayrulewidth=#2pt}
  \providecommand{\huxvb}[2]{\color[RGB]{#1}\vrule width #2pt}
  \providecommand{\huxtpad}[1]{\rule{0pt}{#1}}
  \providecommand{\huxbpad}[1]{\rule[-#1]{0pt}{#1}}

\begin{table}[ht]
\begin{centerbox}
\begin{threeparttable}
 \setlength{\tabcolsep}{0pt}
\begin{tabular}{l l l l l l l}


\hhline{>{\huxb{0, 0, 0}{0.4}}->{\huxb{0, 0, 0}{0.4}}->{\huxb{0, 0, 0}{0.4}}->{\huxb{0, 0, 0}{0.4}}->{\huxb{0, 0, 0}{0.4}}->{\huxb{0, 0, 0}{0.4}}->{\huxb{0, 0, 0}{0.4}}-}
\arrayrulecolor{black}

\multicolumn{1}{!{\huxvb{0, 0, 0}{0.4}}r!{\huxvb{0, 0, 0}{0}}}{\huxtpad{6pt + 1em}\raggedleft \hspace{6pt} \textbf{week} \hspace{6pt}\huxbpad{6pt}} &
\multicolumn{1}{l!{\huxvb{0, 0, 0}{0}}}{\huxtpad{6pt + 1em}\raggedright \hspace{6pt} \textbf{site} \hspace{6pt}\huxbpad{6pt}} &
\multicolumn{1}{r!{\huxvb{0, 0, 0}{0}}}{\huxtpad{6pt + 1em}\raggedleft \hspace{6pt} \textbf{id} \hspace{6pt}\huxbpad{6pt}} &
\multicolumn{1}{l!{\huxvb{0, 0, 0}{0}}}{\huxtpad{6pt + 1em}\raggedright \hspace{6pt} \textbf{treat} \hspace{6pt}\huxbpad{6pt}} &
\multicolumn{1}{r!{\huxvb{0, 0, 0}{0}}}{\huxtpad{6pt + 1em}\raggedleft \hspace{6pt} \textbf{age} \hspace{6pt}\huxbpad{6pt}} &
\multicolumn{1}{l!{\huxvb{0, 0, 0}{0}}}{\huxtpad{6pt + 1em}\raggedright \hspace{6pt} \textbf{sex} \hspace{6pt}\huxbpad{6pt}} &
\multicolumn{1}{r!{\huxvb{0, 0, 0}{0.4}}}{\huxtpad{6pt + 1em}\raggedleft \hspace{6pt} \textbf{twstrs} \hspace{6pt}\huxbpad{6pt}} \tabularnewline[-0.5pt]


\hhline{>{\huxb{0, 0, 0}{0.4}}->{\huxb{0, 0, 0}{0.4}}->{\huxb{0, 0, 0}{0.4}}->{\huxb{0, 0, 0}{0.4}}->{\huxb{0, 0, 0}{0.4}}->{\huxb{0, 0, 0}{0.4}}->{\huxb{0, 0, 0}{0.4}}-}
\arrayrulecolor{black}

\multicolumn{1}{!{\huxvb{0, 0, 0}{0.4}}r!{\huxvb{0, 0, 0}{0}}}{\cellcolor[RGB]{242, 242, 242}\huxtpad{6pt + 1em}\raggedleft \hspace{6pt} 0 \hspace{6pt}\huxbpad{6pt}} &
\multicolumn{1}{l!{\huxvb{0, 0, 0}{0}}}{\cellcolor[RGB]{242, 242, 242}\huxtpad{6pt + 1em}\raggedright \hspace{6pt} 1 \hspace{6pt}\huxbpad{6pt}} &
\multicolumn{1}{r!{\huxvb{0, 0, 0}{0}}}{\cellcolor[RGB]{242, 242, 242}\huxtpad{6pt + 1em}\raggedleft \hspace{6pt} 1 \hspace{6pt}\huxbpad{6pt}} &
\multicolumn{1}{l!{\huxvb{0, 0, 0}{0}}}{\cellcolor[RGB]{242, 242, 242}\huxtpad{6pt + 1em}\raggedright \hspace{6pt} 2 \hspace{6pt}\huxbpad{6pt}} &
\multicolumn{1}{r!{\huxvb{0, 0, 0}{0}}}{\cellcolor[RGB]{242, 242, 242}\huxtpad{6pt + 1em}\raggedleft \hspace{6pt} 65 \hspace{6pt}\huxbpad{6pt}} &
\multicolumn{1}{l!{\huxvb{0, 0, 0}{0}}}{\cellcolor[RGB]{242, 242, 242}\huxtpad{6pt + 1em}\raggedright \hspace{6pt} 1 \hspace{6pt}\huxbpad{6pt}} &
\multicolumn{1}{r!{\huxvb{0, 0, 0}{0.4}}}{\cellcolor[RGB]{242, 242, 242}\huxtpad{6pt + 1em}\raggedleft \hspace{6pt} 32 \hspace{6pt}\huxbpad{6pt}} \tabularnewline[-0.5pt]


\hhline{>{\huxb{0, 0, 0}{0.4}}|>{\huxb{0, 0, 0}{0.4}}|}
\arrayrulecolor{black}

\multicolumn{1}{!{\huxvb{0, 0, 0}{0.4}}r!{\huxvb{0, 0, 0}{0}}}{\huxtpad{6pt + 1em}\raggedleft \hspace{6pt} 2 \hspace{6pt}\huxbpad{6pt}} &
\multicolumn{1}{l!{\huxvb{0, 0, 0}{0}}}{\huxtpad{6pt + 1em}\raggedright \hspace{6pt} 1 \hspace{6pt}\huxbpad{6pt}} &
\multicolumn{1}{r!{\huxvb{0, 0, 0}{0}}}{\huxtpad{6pt + 1em}\raggedleft \hspace{6pt} 1 \hspace{6pt}\huxbpad{6pt}} &
\multicolumn{1}{l!{\huxvb{0, 0, 0}{0}}}{\huxtpad{6pt + 1em}\raggedright \hspace{6pt} 2 \hspace{6pt}\huxbpad{6pt}} &
\multicolumn{1}{r!{\huxvb{0, 0, 0}{0}}}{\huxtpad{6pt + 1em}\raggedleft \hspace{6pt} 65 \hspace{6pt}\huxbpad{6pt}} &
\multicolumn{1}{l!{\huxvb{0, 0, 0}{0}}}{\huxtpad{6pt + 1em}\raggedright \hspace{6pt} 1 \hspace{6pt}\huxbpad{6pt}} &
\multicolumn{1}{r!{\huxvb{0, 0, 0}{0.4}}}{\huxtpad{6pt + 1em}\raggedleft \hspace{6pt} 30 \hspace{6pt}\huxbpad{6pt}} \tabularnewline[-0.5pt]


\hhline{>{\huxb{0, 0, 0}{0.4}}|>{\huxb{0, 0, 0}{0.4}}|}
\arrayrulecolor{black}

\multicolumn{1}{!{\huxvb{0, 0, 0}{0.4}}r!{\huxvb{0, 0, 0}{0}}}{\cellcolor[RGB]{242, 242, 242}\huxtpad{6pt + 1em}\raggedleft \hspace{6pt} 4 \hspace{6pt}\huxbpad{6pt}} &
\multicolumn{1}{l!{\huxvb{0, 0, 0}{0}}}{\cellcolor[RGB]{242, 242, 242}\huxtpad{6pt + 1em}\raggedright \hspace{6pt} 1 \hspace{6pt}\huxbpad{6pt}} &
\multicolumn{1}{r!{\huxvb{0, 0, 0}{0}}}{\cellcolor[RGB]{242, 242, 242}\huxtpad{6pt + 1em}\raggedleft \hspace{6pt} 1 \hspace{6pt}\huxbpad{6pt}} &
\multicolumn{1}{l!{\huxvb{0, 0, 0}{0}}}{\cellcolor[RGB]{242, 242, 242}\huxtpad{6pt + 1em}\raggedright \hspace{6pt} 2 \hspace{6pt}\huxbpad{6pt}} &
\multicolumn{1}{r!{\huxvb{0, 0, 0}{0}}}{\cellcolor[RGB]{242, 242, 242}\huxtpad{6pt + 1em}\raggedleft \hspace{6pt} 65 \hspace{6pt}\huxbpad{6pt}} &
\multicolumn{1}{l!{\huxvb{0, 0, 0}{0}}}{\cellcolor[RGB]{242, 242, 242}\huxtpad{6pt + 1em}\raggedright \hspace{6pt} 1 \hspace{6pt}\huxbpad{6pt}} &
\multicolumn{1}{r!{\huxvb{0, 0, 0}{0.4}}}{\cellcolor[RGB]{242, 242, 242}\huxtpad{6pt + 1em}\raggedleft \hspace{6pt} 24 \hspace{6pt}\huxbpad{6pt}} \tabularnewline[-0.5pt]


\hhline{>{\huxb{0, 0, 0}{0.4}}|>{\huxb{0, 0, 0}{0.4}}|}
\arrayrulecolor{black}

\multicolumn{1}{!{\huxvb{0, 0, 0}{0.4}}r!{\huxvb{0, 0, 0}{0}}}{\huxtpad{6pt + 1em}\raggedleft \hspace{6pt} 8 \hspace{6pt}\huxbpad{6pt}} &
\multicolumn{1}{l!{\huxvb{0, 0, 0}{0}}}{\huxtpad{6pt + 1em}\raggedright \hspace{6pt} 1 \hspace{6pt}\huxbpad{6pt}} &
\multicolumn{1}{r!{\huxvb{0, 0, 0}{0}}}{\huxtpad{6pt + 1em}\raggedleft \hspace{6pt} 1 \hspace{6pt}\huxbpad{6pt}} &
\multicolumn{1}{l!{\huxvb{0, 0, 0}{0}}}{\huxtpad{6pt + 1em}\raggedright \hspace{6pt} 2 \hspace{6pt}\huxbpad{6pt}} &
\multicolumn{1}{r!{\huxvb{0, 0, 0}{0}}}{\huxtpad{6pt + 1em}\raggedleft \hspace{6pt} 65 \hspace{6pt}\huxbpad{6pt}} &
\multicolumn{1}{l!{\huxvb{0, 0, 0}{0}}}{\huxtpad{6pt + 1em}\raggedright \hspace{6pt} 1 \hspace{6pt}\huxbpad{6pt}} &
\multicolumn{1}{r!{\huxvb{0, 0, 0}{0.4}}}{\huxtpad{6pt + 1em}\raggedleft \hspace{6pt} 37 \hspace{6pt}\huxbpad{6pt}} \tabularnewline[-0.5pt]


\hhline{>{\huxb{0, 0, 0}{0.4}}|>{\huxb{0, 0, 0}{0.4}}|}
\arrayrulecolor{black}

\multicolumn{1}{!{\huxvb{0, 0, 0}{0.4}}r!{\huxvb{0, 0, 0}{0}}}{\cellcolor[RGB]{242, 242, 242}\huxtpad{6pt + 1em}\raggedleft \hspace{6pt} 12 \hspace{6pt}\huxbpad{6pt}} &
\multicolumn{1}{l!{\huxvb{0, 0, 0}{0}}}{\cellcolor[RGB]{242, 242, 242}\huxtpad{6pt + 1em}\raggedright \hspace{6pt} 1 \hspace{6pt}\huxbpad{6pt}} &
\multicolumn{1}{r!{\huxvb{0, 0, 0}{0}}}{\cellcolor[RGB]{242, 242, 242}\huxtpad{6pt + 1em}\raggedleft \hspace{6pt} 1 \hspace{6pt}\huxbpad{6pt}} &
\multicolumn{1}{l!{\huxvb{0, 0, 0}{0}}}{\cellcolor[RGB]{242, 242, 242}\huxtpad{6pt + 1em}\raggedright \hspace{6pt} 2 \hspace{6pt}\huxbpad{6pt}} &
\multicolumn{1}{r!{\huxvb{0, 0, 0}{0}}}{\cellcolor[RGB]{242, 242, 242}\huxtpad{6pt + 1em}\raggedleft \hspace{6pt} 65 \hspace{6pt}\huxbpad{6pt}} &
\multicolumn{1}{l!{\huxvb{0, 0, 0}{0}}}{\cellcolor[RGB]{242, 242, 242}\huxtpad{6pt + 1em}\raggedright \hspace{6pt} 1 \hspace{6pt}\huxbpad{6pt}} &
\multicolumn{1}{r!{\huxvb{0, 0, 0}{0.4}}}{\cellcolor[RGB]{242, 242, 242}\huxtpad{6pt + 1em}\raggedleft \hspace{6pt} 39 \hspace{6pt}\huxbpad{6pt}} \tabularnewline[-0.5pt]


\hhline{>{\huxb{0, 0, 0}{0.4}}|>{\huxb{0, 0, 0}{0.4}}|}
\arrayrulecolor{black}

\multicolumn{1}{!{\huxvb{0, 0, 0}{0.4}}r!{\huxvb{0, 0, 0}{0}}}{\huxtpad{6pt + 1em}\raggedleft \hspace{6pt} 16 \hspace{6pt}\huxbpad{6pt}} &
\multicolumn{1}{l!{\huxvb{0, 0, 0}{0}}}{\huxtpad{6pt + 1em}\raggedright \hspace{6pt} 1 \hspace{6pt}\huxbpad{6pt}} &
\multicolumn{1}{r!{\huxvb{0, 0, 0}{0}}}{\huxtpad{6pt + 1em}\raggedleft \hspace{6pt} 1 \hspace{6pt}\huxbpad{6pt}} &
\multicolumn{1}{l!{\huxvb{0, 0, 0}{0}}}{\huxtpad{6pt + 1em}\raggedright \hspace{6pt} 2 \hspace{6pt}\huxbpad{6pt}} &
\multicolumn{1}{r!{\huxvb{0, 0, 0}{0}}}{\huxtpad{6pt + 1em}\raggedleft \hspace{6pt} 65 \hspace{6pt}\huxbpad{6pt}} &
\multicolumn{1}{l!{\huxvb{0, 0, 0}{0}}}{\huxtpad{6pt + 1em}\raggedright \hspace{6pt} 1 \hspace{6pt}\huxbpad{6pt}} &
\multicolumn{1}{r!{\huxvb{0, 0, 0}{0.4}}}{\huxtpad{6pt + 1em}\raggedleft \hspace{6pt} 36 \hspace{6pt}\huxbpad{6pt}} \tabularnewline[-0.5pt]


\hhline{>{\huxb{0, 0, 0}{0.4}}->{\huxb{0, 0, 0}{0.4}}->{\huxb{0, 0, 0}{0.4}}->{\huxb{0, 0, 0}{0.4}}->{\huxb{0, 0, 0}{0.4}}->{\huxb{0, 0, 0}{0.4}}->{\huxb{0, 0, 0}{0.4}}-}
\arrayrulecolor{black}
\end{tabular}
\end{threeparttable}\par\end{centerbox}

\end{table}
 

\subsection{Introduction}\label{introduction}

The purpose of this project is to examine the effects of botulism toxin
type B (BotB) to treat cervical dystonia over time. Cervical dystonia
(CD) is a chronic neurological disorder, in which patients have painful
involuntary contractions in neck muscles. CD is more prevalent in women
(Jankovic et al., 2023). The prevalence of CD is estimated to be 28-183
cases per million. The data comes from a multicenter randomized clinical
trial for cervical dystonia patients with 9 U.S. sites. Botulism toxin
types A and B are first-line treatments for CD (Wetmore et al., 2025).
The treatment groups included in the study were placebo, 5000 U BotB,
and 10000 U BotB. The response variable is Toronto Western Spasmodic
Torticollis Rating Scale (TWSTRS) total score, which ranges from 0 to 85
and is comprised of disability (0-30), pain (0-20), and severity (0-35)
subscores. Only total score is included in this data.The TWSTRS score
was measured at week 0 (baseline), and 2,4,6,8,12, and 16 weeks after
treatment start. Site is included in the dataset but no further details
about site were included in the available dataset documentation.

\subsection{Methods}\label{methods}

\subsubsection{\texorpdfstring{\emph{Study
Population}}{Study Population}}\label{study-population}

Inclusion and exclusion criteria for the trial were not available.
Duration of CD and age at onset were not known. The study included 109
patients (67 (61\%)) females.The mean age was 56 (12). Median age was 56
years. The mean TWSTRS score at baseline was 46 (10). It was not known
if the patients received prior BotB treatments. Information about the
randomization schedule was not provided.

\subsubsection{\texorpdfstring{\emph{Statistical
Analyses}}{Statistical Analyses}}\label{statistical-analyses}

Number of observations, mean, median, standard deviation (SD), minimum
(min) and maximum (max) were provided for age. Mean and SD were
calculated for TWSTRS score at baseline. Frequencies and percentages
were reported for categorical variables. GLM, GLMM, and GEE models were
fit using TWSTRS total score as the response variable and blank, blank,
blank as covariates.

\subsection{Results}\label{results}

\begin{longtable}[t]{lccccc}
\caption{\label{tab:unnamed-chunk-3}Summary of Demographic and Baseline Characteristics}\\
\toprule
\multicolumn{3}{c}{ } & \multicolumn{2}{c}{\textbf{BotB}} & \multicolumn{1}{c}{ } \\
\cmidrule(l{3pt}r{3pt}){4-5}
\textbf{Characteristic} & \makecell[c]{\textbf{Overall}\ \ \\N = 109} & \makecell[c]{\textbf{Placebo}\ \ \\N = 36} & \makecell[c]{\textbf{5000 U}\ \ \\N = 36} & \makecell[c]{\textbf{10000 U}\ \ \\N = 37} & \textbf{p-value}\\
\midrule
Sex &  &  &  &  & 0.0706\\
\hspace{1em}Female & 67 (61\%) & 21 (58\%) & 18 (50\%) & 28 (76\%) & \\
\hspace{1em}Male & 42 (39\%) & 15 (42\%) & 18 (50\%) & 9 (24\%) & \\
Age (years) &  &  &  &  & 0.6198\\
\hspace{1em}No. obs. & 109 & 36 & 36 & 37 & \\
\hspace{1em}Mean (SD) & 56 (12) & 54 (12) & 57 (12) & 56 (12) & \\
\hspace{1em}Median & 56 & 56 & 57 & 54 & \\
\hspace{1em}Min, Max & 26, 83 & 26, 79 & 35, 83 & 34, 76 & \\
TWSTRS total score at baseline &  &  &  &  & 0.3307\\
\hspace{1em}Mean (SD) & 46 (10) & 44 (9) & 46 (10) & 47 (10) & \\
\bottomrule
\multicolumn{6}{l}{\rule{0pt}{1em}\textsuperscript{1} BotB = botulinum toxin type B; TWSTRS = Toronto Western Spasmodic Torticollis Rating Scale.}\\
\multicolumn{6}{l}{\rule{0pt}{1em}\textsuperscript{2} Pearson's Chi-squared test; Kruskal-Wallis rank sum test}\\
\end{longtable}

\subsubsection{Generalized Linear Model
(GLM)}\label{generalized-linear-model-glm}

A generalized linear model (GLM model) using a Gaussian link was fit
including week, site, treatment, age, and sex as predictors (Table 2).

\begin{longtable}[]{@{}lrrrr@{}}
\caption{GLM Model Summary (Gaussian Link)}\tabularnewline
\toprule\noalign{}
term & estimate & std.error & statistic & p.value \\
\midrule\noalign{}
\endfirsthead
\toprule\noalign{}
term & estimate & std.error & statistic & p.value \\
\midrule\noalign{}
\endhead
\bottomrule\noalign{}
\endlastfoot
(Intercept) & 32.166 & 2.767 & 11.624 & 0.000 \\
week & 0.234 & 0.080 & 2.923 & 0.004 \\
treat5000 U & -0.030 & 1.120 & -0.027 & 0.978 \\
treat10000 U & 0.287 & 1.127 & 0.254 & 0.799 \\
age & 0.072 & 0.039 & 1.854 & 0.064 \\
sexMale & -1.938 & 1.006 & -1.927 & 0.054 \\
site2 & 13.216 & 1.860 & 7.107 & 0.000 \\
site3 & -2.183 & 1.933 & -1.130 & 0.259 \\
site4 & 4.635 & 2.132 & 2.174 & 0.030 \\
site5 & 7.403 & 2.413 & 3.068 & 0.002 \\
site6 & 9.617 & 1.849 & 5.201 & 0.000 \\
site7 & 1.862 & 1.912 & 0.974 & 0.331 \\
site8 & -2.493 & 1.773 & -1.406 & 0.160 \\
site9 & 10.198 & 2.011 & 5.071 & 0.000 \\
\end{longtable}

The coefficient for week was statistically significant, indicating an
overall linear trend in TWSTRS score over time. There was notable
variation across study sites, with several sites showing significantly
higher TWSTRS scores compared to the reference site, highlighting
site-level differences. However, because the GLM assumes independence of
observations, the repeated measures within individuals violate this
assumption. As a result, the standard errors and p-values may be
underestimated, and inference should be interpreted with caution. This
motivates the subsequent use of correlation-aware models such as GEE and
GLMM.

\subsubsection{GLM Model Diagnostics}\label{glm-model-diagnostics}

Diagnostics were assessed for the GLM model.

\pandocbounded{\includegraphics[keepaspectratio]{BIOS-667-Group-Project_files/figure-latex/unnamed-chunk-5-1.pdf}}

Figure 1 shows the residuals versus the fitted values. The absence of
strong patterns or fanning suggests that the linearity and
homoscedasticity assumptions are reasonably met.

\pandocbounded{\includegraphics[keepaspectratio]{BIOS-667-Group-Project_files/figure-latex/unnamed-chunk-6-1.pdf}}

Figure 2 presents a QQ plot of the residuals, which largely follow the
45 degree reference line with mild deviations in the tails. This
indicates that the normality assumption is approximately satisfied.

\pandocbounded{\includegraphics[keepaspectratio]{BIOS-667-Group-Project_files/figure-latex/unnamed-chunk-7-1.pdf}}

Finally, figure 3 displays Cook's distance for all observations. The
conventional threshold, 4/(n-k-1), where n is the number of observations
and k is the number of predictors, was used to assess the influence of
the observations. Several observations exceeded the threshold and were
considered potentially influential and warranted further investigation.
Therefore, a second GLM was fit, excluding those observations.

\begin{longtable}[]{@{}lrrrr@{}}
\caption{GLM Model Summary (Excluding Influential
Observations)}\tabularnewline
\toprule\noalign{}
term & estimate & std.error & statistic & p.value \\
\midrule\noalign{}
\endfirsthead
\toprule\noalign{}
term & estimate & std.error & statistic & p.value \\
\midrule\noalign{}
\endhead
\bottomrule\noalign{}
\endlastfoot
(Intercept) & 32.882 & 2.478 & 13.272 & 0.000 \\
week & 0.215 & 0.072 & 3.002 & 0.003 \\
treat5000 U & -0.185 & 1.004 & -0.184 & 0.854 \\
treat10000 U & 0.800 & 1.001 & 0.799 & 0.424 \\
age & 0.050 & 0.035 & 1.449 & 0.148 \\
sexMale & -1.046 & 0.903 & -1.158 & 0.247 \\
site2 & 13.405 & 1.667 & 8.043 & 0.000 \\
site3 & -1.417 & 1.741 & -0.814 & 0.416 \\
site4 & 7.636 & 1.947 & 3.921 & 0.000 \\
site5 & 10.761 & 2.283 & 4.714 & 0.000 \\
site6 & 10.152 & 1.661 & 6.111 & 0.000 \\
site7 & 2.380 & 1.707 & 1.394 & 0.164 \\
site8 & -3.384 & 1.605 & -2.108 & 0.035 \\
site9 & 10.049 & 1.861 & 5.400 & 0.000 \\
\end{longtable}

Excluding influential observations resulted in modest shifts in the
parameter estimates and slightly improved precision. Notably, the effect
of week remained statistically significant and variation across study
sites continued to be pronounced. This underscores that site-level
differences remain important even after removing influential
observation.s

Overall, excluding influential points slightly adjusted the estimates
and improved precision, but the main patterns of association were
consistent with the original GLM. As with the previous model, the GLM
assumptions are not appropriate for longitudinal data with correlated
repeated measures. The following GEE and GLMM analyses address this
limitation by explicitly modeling within-subject correlation and random
effects.

\subsubsection{Generalized Estimating Equation
(GEE)}\label{generalized-estimating-equation-gee}

A GEE model with Gaussian family, identity link, and an exchangeable
correlation structure. The mean model included week, treatment,
treatment-by-week interaction, and baseline age and sex, with Placebo as
the reference treatment group.

\pandocbounded{\includegraphics[keepaspectratio]{BIOS-667-Group-Project_files/figure-latex/unnamed-chunk-10-1.pdf}}

Figure 4 shows Pearson residuals plotted against the fitted TWSTRS
values with points colored by treatment group. The residuals are
centered around zero and do not show a strong funnel shape, supporting
the constant-variance assumption. However, many residuals fall outside
\(\pm2\), indicating wide individual variation in TWSTRS trajectories
and suggesting that the simple Gaussian working model does not fully
capture the variance structure. Even so, because the GEE analysis used
robust standard errors, the uncertainty in the regression coefficients
is driven by the observed variability of the residuals within subjects,
rather than depending on the working variance and correlation.

\pandocbounded{\includegraphics[keepaspectratio]{BIOS-667-Group-Project_files/figure-latex/unnamed-chunk-11-1.pdf}}

Figure 5 shows mean TWSTRS over time by treatment group. All groups show
decreases from baseline through weeks 2--4, suggesting early
improvement, followed by increases after week 4. Around week 10, both
active treatment groups begin to show higher TWSTRS scores than the
placebo group.

\begin{longtable}[]{@{}lrrrr@{}}
\caption{GEE Model Summary}\tabularnewline
\toprule\noalign{}
term & estimate & std.error & statistic & p.value \\
\midrule\noalign{}
\endfirsthead
\toprule\noalign{}
term & estimate & std.error & statistic & p.value \\
\midrule\noalign{}
\endhead
\bottomrule\noalign{}
\endlastfoot
(Intercept) & 41.192 & 4.612 & 79.769 & 0.000 \\
week & 0.097 & 0.114 & 0.731 & 0.392 \\
treat5000 U & -0.810 & 2.758 & 0.086 & 0.769 \\
treat10000 U & -2.738 & 2.372 & 1.333 & 0.248 \\
age & 0.014 & 0.078 & 0.033 & 0.855 \\
sexMale & -2.494 & 2.117 & 1.388 & 0.239 \\
week:treat5000 U & 0.115 & 0.133 & 0.745 & 0.388 \\
week:treat10000 U & 0.323 & 0.146 & 4.899 & 0.027 \\
\end{longtable}

Table 4 summarizes the GEE estimates for TWSTRS over time by treatment,
adjusted for age and sex. The main effects for treat5000U (−0.810, p =
0.769) and treat10000U (−2.738, p = 0.248) compare baseline TWSTRS at
week 0 with placebo. Both estimates are negative but not statistically
significant, indicating no clear baseline differences in mean TWSTRS
between treatment groups and placebo. The treatment-by-week interaction
terms describe how each treatment's trajectory differs from placebo over
time. For 5000U, the additional slope relative to placebo is small and
not significant (0.115, p = 0.388), suggesting that its average linear
trend over time is similar to placebo. For 10000U, the additional slope
is larger and statistically significant (0.323, p = 0.027), indicating
an average increase in TWSTRS over 16 weeks compared with placebo.
Because higher TWSTRS scores reflect worse symptoms, these results imply
that low-dose regimen behaves similarly to placebo, while the high-dose
regimen is associated with faster worsening over time. Overall, the GEE
model provides a reasonable approach for assessing whether treatment
affects population-average TWSTRS trajectories over time, while
recognizing that the true time course is somewhat non-linear and that
estimated treatment effects are relatively small.

\subsection{Discussion}\label{discussion}

Summarization of main points, conclusions based in results Summarization
of the various models applied, which ones you prefer and base your
interpretations off of and why Discussion of limitations if any

\subsection*{References}\label{references}
\addcontentsline{toc}{subsection}{References}

\phantomsection\label{refs}
\begin{CSLReferences}{1}{0}
\bibitem[\citeproctext]{ref-jankovic_treatment_2023}
Jankovic, J., Tsui, J., \& Brin, M. F. (2023). Treatment of cervical
dystonia with {Botox} ({Onabotulinumtoxina}): {Development}, insights,
and impact. \emph{Medicine}, \emph{102}(S1), e32403.
\url{https://doi.org/10.1097/MD.0000000000032403}

\bibitem[\citeproctext]{ref-wetmore_clinical_2025}
Wetmore, E., Roberts, H., Livinski, A. A., Camacho, T., Eaton, C.,
Norato, G., Hallett, M., \& Stacy, M. (2025). Clinical response to
placebo botulinum toxin injection in cervical dystonia---a systematic
review and meta-analysis. \emph{Dystonia}, \emph{4}, 14297.
\url{https://doi.org/10.3389/dyst.2025.14297}

\end{CSLReferences}

\end{document}
